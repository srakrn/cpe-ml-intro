% !TEX program = xelatex

\documentclass[]{beamer}

\usepackage{listings}
\usepackage{color}

\definecolor{codegreen}{rgb}{0,0.6,0}
\definecolor{codegray}{rgb}{0.5,0.5,0.5}
\definecolor{codepurple}{rgb}{0.58,0,0.82}
\definecolor{backcolour}{rgb}{0.95,0.95,0.92}

\lstdefinestyle{codeblock}{
    backgroundcolor=\color{backcolour},   
    commentstyle=\color{codegreen},
    keywordstyle=\color{magenta},
    numberstyle=\tiny\color{codegray},
    stringstyle=\color{codepurple},
    basicstyle=\footnotesize,
    breakatwhitespace=false,         
    breaklines=true,                 
    captionpos=b,                    
    keepspaces=true,                 
    numbers=left,                    
    numbersep=5pt,                  
    showspaces=false,                
    showstringspaces=false,
    showtabs=false,                  
    tabsize=4
}
\lstset{style=codeblock}

\usepackage{smartdiagram}

\title{Introduction to Machine Learning}
\subtitle{Knowledge Sharing for CPE/SKE students}
\author{Sirakorn Lamyai}
\institute{Student, Kasetsart U.}
\date{\today}
\begin{document}

\begin{frame}
\titlepage
\end{frame}

\begin{frame}
\frametitle{Outline}
\tableofcontents
\end{frame}

\section{Introduction to Machine Learning}

\subsection{What is Machine Learning?}

\begin{frame}
\frametitle{What is Machine Learning?}
\pause
\begin{figure}
\includegraphics[scale=1]{imgs/recaptcha.png}
\end{figure}
\begin{itemize}
\pause
\item This is Recaptcha.
\begin{itemize}
\pause
\item Recaptcha helps stop millions of spam a day.
\pause
\item In some old days, we have to type Captcha texts to distinguish ourself from bots.
\pause
\item How is it possible that with a single click, an automated system can distinguish bots from humans?
\end{itemize}
\end{itemize}
\end{frame}

\subsubsection{Traditional programming approach}

\begin{frame}
\frametitle{Traditional programming approach}
\begin{center}
\smartdiagram[circular diagram]{Analyse, Algorithm, Test, Improve, Repeat}
\end{center}
\end{frame}


\subsubsection{Machine learning approach}

\begin{frame}
\frametitle{Machine learning approach}
\begin{center}
\smartdiagram[circular diagram]{Analyse, Machine Learning, Validation, Improve, Repeat}
\end{center}
\end{frame}

\begin{frame}
\frametitle{In other words...}
\begin{center}
Machine Learning \\
\onslide<2-> \huge = Data + Data analysis algorithm \\
\onslide<3-> \Huge = Adapt to change
\end{center}
\end{frame}

\section{Types of Machine Learning Problems}

\begin{frame}
\frametitle{Types of Machine Learning problems}
\begin{columns}
\column{0.5\textwidth}
\begin{enumerate}
\item<2-> Supervised learning
\item<3-> Unsupervised learning
\item<4-> Reinforcement learning
\end{enumerate}
\pause
\column<5->{0.5\textwidth}
Determined by \\
\Huge Labels
\end{columns}
\end{frame}

\subsection{Supervised learning}

\begin{frame}
\frametitle{Supervised learning}
\begin{figure}
\includegraphics[scale=.35]{imgs/supervised_learning.png}
\end{figure}
\end{frame}

\subsection{Unsupervised learning}

\begin{frame}
\frametitle{Unsupervised learning}
\begin{figure}
\includegraphics[scale=.3]{imgs/kmeans.png}
\end{figure}
\end{frame}

\subsection{Reinforcement learning}

\begin{frame}
\frametitle{Reinforcement learning}
\begin{figure}
\includegraphics[scale=.7]{imgs/reinforcement_learning.png}
\end{figure}
\end{frame}

\section{Model}

\begin{frame}
\begin{center}
\Huge Model
\end{center}
\end{frame}

\begin{frame}
\frametitle{Model}
\begin{itemize}
\item<2-> A result of the combination between...
\begin{itemize}
\item<3-> a \textbf{method} to recognise the data, and
\item<4-> \textbf{sample datas} for such the method
\end{itemize}
\end{itemize}
\end{frame}

\begin{frame}
\frametitle{Model}
\begin{columns}
\column<2->{0.5\textwidth}
\begin{figure}
\includegraphics[scale=.3]{imgs/simple_knn.png}
\end{figure}
\column<3->{0.5\textwidth}
Determine which group should the purple dot be in (red/green/blue) by \textbf{checking the colour of its nearest dot.}
\end{columns}
\begin{columns}
\column<4->{0.5\textwidth}
\begin{center}
\Large Data
\end{center}
\column<5->{0.5\textwidth}
\begin{center}
\Large Method
\end{center}
\end{columns}
\end{frame}

\subsection{A good model}
\begin{frame}
\begin{center}
{\Huge Good model?}\\
\end{center}
\end{frame}

\begin{frame}
\frametitle{Good model}
\begin{figure}
\includegraphics[scale=.4]{imgs/linreg_1.png}
\end{figure}
\begin{center}
How should we \textit{draw} the line to predict this data?
\end{center}
\end{frame}

\begin{frame}
\frametitle{Good model}
\begin{figure}
\includegraphics[scale=.4]{imgs/linreg_2.png}
\end{figure}
\begin{center}
Blue, red, or green line?
\end{center}
\end{frame}

\subsubsection{Overfitting and underfitting}
\begin{frame}
\frametitle{Overfitting and underfitting}
\begin{columns}
\column{0.5\textwidth}
\begin{figure}
\includegraphics[scale=.3]{imgs/linreg_2.png}
\end{figure}
\column{0.5\textwidth}
\begin{enumerate}
\item<2-> Underfitting
\begin{itemize}
\item<3-> Our model \textbf{fails to know the data's trends}
\item<4-> Resulting in failure to predict further data
\end{itemize}
\item<5-> Overfitting
\begin{itemize}
\item<6-> Our model \textbf{memorise instead of generalise}
\item<7-> Resulting in failure to catch the trend
\end{itemize}
\end{enumerate}
\end{columns}
\end{frame}

\begin{frame}
\begin{center}
{\LARGE Good model}\\
{\onslide<2-> \Huge{must generalise}}
\end{center}
\end{frame}

\end{document}